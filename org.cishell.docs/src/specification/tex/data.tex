\section{Data Specification}
\label{dataSpec}
\subsection*{\textit{Version 1.0}}
\subsection{Introduction}

Data to be operated on is passed around in \class{Data} objects which hold the
real data, the data's format, and its properties (metadata). The real data can be
any Java \class{Object}. The format is a string describing the 'format' of the
data, see next section. Finally, the properties help describe the data. The label
to give the data, the parent \class{Data} object from which it was derived from,
and a coarse data type can all be defined in the \class{Data}'s properties. See
the \class{DataProperty} interface definition for specific properties to use.

\subsection{Data Format Specification}

Data formats are used by algorithms specifying what \class{Data} items are
consumed/producted and \class{Data} objects must know what the format of its
contained real data is. This format is simply a string that says what the
format of the real data is. If the real data is a \class{java.io.File}, use a
MIME type\footnote{If no official mime type is available for a file format, a
made up one can be used, but must still conform to how mime types are
constructed, see RFCs 3023 (http://tools.ietf.org/html/rfc3023) and 4288
(http://tools.ietf.org/html/rfc4288).} prefixed by ``file:'', i.e.,
``file:\textit{mime/type}''. If the real data is a \class{java.io.File} known
only by file extension (only applicable for validator algorithms), use the format
``file-ext:\textit{file-extension}''. Otherwise, if the real data is a Java
\class{Object}, use the full Java class as a string, i.e., ``java.lang.String''
or ``my.package.SpecialClass''.

To specify that a single \class{Data} object contains one or more
sub-\class{Data} objects of a single format, prefix the type with a ``+''. For
zero or more, prefix the type with a ``*''. This corresponds to a \class{Data}
object that is wrapping multiple (zero or one or more depending on the prefix)
other \class{Data} objects of the associated format in a Java array
(\class{Data[]}) stored as its wrapped real data. This is useful for algorithms
that can work on a variable number of \class{Data} items.
