\section{User Adjustable Preferences Specification}
\label{userPrefsSpec}
\subsection*{\textit{Version 1.0}}
\subsection{Introduction}

The user-adjustable preferences specification defines how any service can publish
user-adjustable preferences both globally and locally. In addition to global and
local preferences, algorithms can allow the system to allow end-users to adjust
the default values for algorithms' user-entered input parameters specification
published to the \class{MetaTypeService}. For storing data that is not directly
end-user adjustable, see chapter \ref{preferencesService}.

\subsection{Publishing User Adjustable Preferences}

\subsubsection*{Create an ObjectClassDefinition (OCD)} To define parameters that
can be adjusted by an end-user, an algorithm developer must first create an
\class{ObjectClassDefinition} which details the parameters to be published. This
OCD must be visible to the \class{MetaTypeService} either through the use of a
METADATA.XML file or by the service implementing \class{MetaTypeProvider} and
\class{ManagedService}. See section \ref{GUISpec} for more information.

\subsubsection*{Designate an OCD a Persistent ID (PID)} Then they must designate
the \class{ObjectClassDefinition} a persistent id (PID). The PID can be
designated in two ways. The simplest way is by following the convention of
creating a string with the associated service's ``service.pid'' and appending
either ``.prefs.local'' or ``.prefs.global''. The other way is to designate
whatever PID the developer wishes and to provide a service property
``local\_pref\_pid'' or ``global\_pref\_pid'' which is set to whatever PID they
chose.

\subsubsection*{Declare What Preferences are to be Published} To let the system
know that you wish to publish preferences, the system properties must contain a
``prefs\_published'' key with zero or more of the following values (separated by
commas): ``local'' for publishing local prefs, ``global'' for global prefs, and
``param-defaults'' for algorithm parameter defaults.

\subsubsection*{Algorithm Parameter Defaults} By publishing algorithm parameter
defaults, algorithm developers allow end-users to adjust the default values they
see when running their algorithm. This is typically accomplished by wrapping the
\class{MetaTypeProvider} published by the algorithm to the
\class{MetaTypeService} with overridden \class{AttributeDefinitions} that
change their default value. Many systems will have this on by default, but if
the ``prefs\_published'' key is set in the algorithm's service metadata and
``param-defaults'' is not set, then this feature will be disabled for the
algorithm.

\subsubsection*{Receiving Preference Data} To be notified of changes to local or
global preferences, the service must implement
\class{org.\-osgi.\-service.\-cm.\-ManagedService} and set in their service
metadata ``receive\_prefs=true''. When either the local or global preferences are updated,
the updated method will be passed a \class{Dictionary} of all of the id/value
pairs, including the updated ones. Local preferences will have the same ids as
the \class{AttributeDefinition}s (AD) defined in the associated OCD. The local
preferences will also have an additional id ``Bundle-Version", which contains the
version of the service's associated bundle that was used when the preference data
was last updated. Global preferences will have the same ids (plus a
``Bundle-Version'' id analagous to local preference's) from their OCD's ADs
prefixed by the PID of the published global preference. In this way, all global
preferences published in the system will be available to anyone receiving
preference data. Note that one does not have to publish any preferences to
receive just global preference data.
