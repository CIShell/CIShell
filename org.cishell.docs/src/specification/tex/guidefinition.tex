%% Cyberinfrastructure Shell (CIShell) Core Specification
%%
%% Copyright 2006,2007,2008 Indiana University
%%
%% Licensed under the Apache License, Version 2.0 (the "License");
%% you may not use this file except in compliance with the License.
%% You may obtain a copy of the License at
%%
%%     http://www.apache.org/licenses/LICENSE-2.0
%%
%% Unless required by applicable law or agreed to in writing, software
%% distributed under the License is distributed on an "AS IS" BASIS,
%% WITHOUT WARRANTIES OR CONDITIONS OF ANY KIND, either express or implied.
%% See the License for the specific language governing permissions and
%% limitations under the License.
%%
%

\section{User Interface Specification}
\label{GUISpec}
\subsection*{\textit{Version 1.0}}
\subsection{Introduction}

For many algorithms, just looking at the data given isn't enough. Additional
input parameters are often needed to know how to operate on a given piece of
data. An algorithm can define what parameters are needed by providing a
\class{MetaTypeProvider}. It defines the types, value range, and textual
description of the parameters needed. From this information, a user interface
(UI) can be created that asks a user for the data. The \class{MetaTypeProvider}
is not tied to any specific UI, so it can be reused depending on the context
(desktop application, web application, command line, etc.).

\class{MetaTypeProvider} is a Java interface defined in the OSGi R4 Specification
Service Compendium as part of the ``Metatype Service Specification,'' section
105. A \class{MetaTypeProvider} can be thought of as a collection of UIs. Each UI
is called an \class{ObjectClassDefinition}, which provides a UI name and
description and is a collection of parameters. Each parameter is an
\class{AttributeDefinition} which includes the type, label, description, default
value, and range of valid values. Drop-down boxes can also be defined by using
option labels and values with the \class{AttributeDefinition}. OSGi's
documentation should be consulted for more information.

\subsection{MetaTypeProvider Extensions}

Some minor extensions to \class{MetaTypeProvider} were made to support some use
cases. The \class{MetaTypeProvider} supports several primitive types such as
strings, integers, booleans, etc, but several useful types are missing. To
support more types, an \class{AttributeDefinition} (AD) of type ``string'' has
its default value set to a certain string so that the UI builder recognizes this
and selects an appropriate widget. When the algorithm receives the user-entered
parameters, the associated value will be of type \class{java.lang.String}, but
should contain the correct value as defined below.

\subsubsection*{file:}
An AD with type ``string'' and default value ``file:'' will receive a string
pointing to the absolute path to the file selected by the end-user.

\subsubsection*{directory:}
An AD with type ``string'' and default value ``directory:'' will receive a string
pointing to the absolute path to the directory selected by the end-user.

\subsubsection*{password:}
An AD with type ``string'' and default value ``password:'' will receive a string
corresponding to the entered password.

\subsubsection*{rgb:} 
An AD with type ``string'' and default value ``rgb:'' will
receive a string which is a comma separated list that corresponds to the RGB
color values the user chose. Each item in the comma separated list would be
between 0 and 255. The first item would be the red value, second green value,
and third blue value.

\subsection{Publishing MetaTypeProviders}

For user-adjustable preferences and algorithm input parameters, a
\class{MetaTypeProvider} is required to be published to the
\class{MetaTypeService}. This can be done in two ways, through code or by a
METADATA.XML file.

To publish through code, a few steps must be followed. First, the service to be
registered with the OSGi service registry must fully implement
\class{org.\-osgi.\-service.\-cm.\-ManagedService} and
\class{org.\-osgi.\-service.\-metatype.\-MetaTypeProvider}. Second, when
registering the service, both \class{ManagedService} and \class{MetaTypeProvider}
must be in the list of interfaces the service implements. If these two things are
done, the \class{MetaTypeService} will notice it and add it to its registry of
\class{MetaTypeProviders}.

The recommended way to publish \class{MetaTypeProviders} is to publish through a
METADATA.XML file. A METADATA.XML file must be included in the algorithm's OSGi
bundle in a specific directory, ``OSGI-INF/metatype/''. The
\class{MetaTypeService} will notice this in the bundle and add it to its registry
of \class{MetaTypeProviders}. See the ``Metatype Service Specification'' in the
OSGi specification for details on the XML format.
