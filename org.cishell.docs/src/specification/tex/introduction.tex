\chapter{Introduction}

The Cyberinfrastructure Shell (CIShell) is an open source, community-driven
platform for the integration and utilization of datasets, algorithms, tools, and
computing resources. It is built specifically to enable (1) algorithm developers
to write and disseminate their algorithms in their favorite programming language
while retaining their intellectual rights after distribution; (2) data holders to
easily disseminate their data for use by others; (3) application developers to
design applications from custom sets of algorithms and datasets that interoperate
seamlessly; and finally (4) researchers, educators, and practitioners to use
existing datasets and algorithms to further science.

\section{CIShell Platform Overview}

The CIShell Platform consists of Java interface definitions of algorithms, data,
services for algorithm developers, and services for application developers. Much
of the platform uses metadata and is fully defined.

\section{What is New}

This is the first release of the CIShell Platform Specification. Future
versions will strive for backwards compatibility.

\section{Reader Level}

This specification is written for the following audiences:
\begin{itemize}
  \item Java Algorithm developers
  \item Non-Java Algorithm developers
  \item Framework and system service developers (system developers)
  \item Application developers building on CIShell
\end{itemize}

The CIShell Specifications assume that the reader has at least one year of
practical experience in writing Java programs. CIShell is built to run on the
OSGi Service Platform Release 4 and thus a working knowledge of OSGi is expected.
OSGi (and thus CIShell) is highly dynamic and must be taken into consideration
when developing anything on CIShell.

Non-Java Algorithm developers may not need to know any Java and should be mainly
concerned with the metadata definitions for algorithms and data. They may also
need to be aware of OSGi and the other services CIShell provides, but more than
likely will not directly interact with them.

\section{Conventions and Terms}

The conventions and terms are exactly the same as from OSGi's Core
Specification, section 1.4.

\section{Version Information}

This is the first release of the CIShell Platform Specification. All packages are
at 1.0 for this release. Subsequent releases may increase the version number of
specific packages, if changes have been made.

\begin{table}[h!]
\begin{tabular}{l l l}
\textbf{Item} & \textbf{Package} & \textbf{Version} \\
Framework Specification & org.cishell.framework & Version 1.0 \\
Algorithm Specification & org.cishell.framework.algorithm & Version 1.0 \\
Data Specification & org.cishell.framework.data & Version 1.0 \\
Data Conversion Service Specification & org.cishell.service.conversion &
Version 1.0 \\
GUI Builder Service Specification & org.cishell.service.guibuilder & Version
1.0 \\
Log Service Specification & org.osgi.service.log & Version 1.3 \\
Preferences Service Specification & org.osgi.service.prefs & Version 1.1 \\
GUI Preferences Service Specification & org.cishell.service.\comments{TBD} &
Version 1.0 \\
Data Manager Application Service Specification & org.cishell.app.datamanager &
Version 1.0 \\
Scheduler Application Service Specification & org.cishell.app.scheduler &
Version 1.0 \\
\end{tabular}
\caption{Packages and Versions}
\label{table:packageVersions}
\end{table}