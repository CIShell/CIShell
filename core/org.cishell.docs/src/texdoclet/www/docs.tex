\documentclass[12pt]{report}
\def\bl{\mbox{}\newline\mbox{}\newline{}}
\usepackage{ifthen}
\newcommand{\hide}[2]{
\ifthenelse{\equal{#1}{inherited}}%
{}%
{}%
}
\newcommand{\entityintro}[3]{%
  \hbox to \hsize{%
    \vbox{%
      \hbox to .2in{}%
    }%
    {\bf #1}%
    \dotfill\pageref{#2}%
  }
  \makebox[\hsize]{%
    \parbox{.4in}{}%
    \parbox[l]{5in}{%
      \vspace{1mm}\it%
      #3%
      \vspace{1mm}%
    }%
  }%
}
\newcommand{\isep}[0]{%
\setlength{\itemsep}{-.4ex}
}
\newcommand{\sld}[0]{%
\setlength{\topsep}{0em}
\setlength{\partopsep}{0em}
\setlength{\parskip}{0em}
\setlength{\parsep}{-1em}
}
\newcommand{\headref}[3]{%
\ifthenelse{#1 = 1}{%
\addcontentsline{toc}{section}{\hspace{\qquad}\protect\numberline{}{#3}}%
}{}%
\ifthenelse{#1 = 2}{%
\addcontentsline{toc}{subsection}{\hspace{\qquad}\protect\numerline{}{#3}}%
}{}%
\ifthenelse{#1 = 3}{%
\addcontentsline{toc}{subsubsection}{\hspace{\qquad}\protect\numerline{}{#3}}%
}{}%
\label{#3}%
\makebox[\textwidth][l]{#2 #3}%
}%
\newcommand{\membername}[1]{{\it #1}\linebreak}
\newcommand{\divideents}[1]{\vskip -1em\indent\rule{2in}{.5mm}}
\newcommand{\refdefined}[1]{
\expandafter\ifx\csname r@#1\endcsname\relax
\relax\else
{$($ in \ref{#1}, page \pageref{#1}$)$}
\fi}
\newcommand{\startsection}[4]{
\gdef\classname{#2}
\subsection{\label{#3}{\bf {\sc #1} #2}}{
\rule[1em]{\hsize}{4pt}\vskip -1em
\vskip .1in 
#4
}%
}
\newcommand{\startsubsubsection}[2]{
\subsubsection{\sc #1}{%
\rule[1em]{\hsize}{2pt}%
#2}
}
\usepackage{color}
\date{\today}
\pagestyle{myheadings}
\addtocontents{toc}{\protect\def\protect\packagename{}}
\addtocontents{toc}{\protect\def\protect\classname{}}
\markboth{\protect\packagename -- \protect\classname}{\protect\packagename -- \protect\classname}
\oddsidemargin 0in
\evensidemargin 0in
% \topmargin -.8in
\chardef\bslash=`\\
\textheight 9.4in
\textwidth 6.5in
\begin{document}
\sloppy
\raggedright
\tableofcontents
\gdef\packagename{}
\gdef\classname{}
\newpage
\def\packagename{org.wonderly.doclets}
\chapter{\bf Package org.wonderly.doclets}{
\vskip -.25in
\hbox to \hsize{\it Package Contents\hfil Page}
\rule{\hsize}{.7mm}
\vskip .13in
\hbox{\bf Interfaces}
\entityintro{ClassFilter}{l0}{This interface can be implemented and a class name provided
  to the doclet to filter which classes are and are not included
  in the output document.}
\vskip .13in
\hbox{\bf Classes}
\entityintro{Package}{l1}{This class is used to manage the contents of a Java package.}
\entityintro{TableInfo}{l2}{This class provides support for converting HTML tables into LaTeX tables.}
\entityintro{TexDoclet}{l3}{This class provides a Java 2, {\tt javadoc} Doclet which generates
  a LaTeX2e document out of the java classes that it is used on.}
\vskip .1in
\rule{\hsize}{.7mm}
\vskip .1in
\newpage
\section{Interfaces}{
\startsection{Interface}{ClassFilter}{l0}{%
{\small This interface can be implemented and a class name provided
  to the doclet to filter which classes are and are not included
  in the output document.}
\vskip .1in 
\startsubsubsection{Declaration}{
\fbox{\vbox{
\hbox{\vbox{\small public interface 
ClassFilter}}
}}}
\startsubsubsection{Methods}{
\vskip -2em
\begin{itemize}
\item{\vskip -1.9ex 
\membername{includeClass}
{\tt public boolean {\bf includeClass}( {\tt com.sun.javadoc.ClassDoc } {\bf cd} )
\label{l4}\label{l5}}%end signature
\begin{itemize}
\sld
\item{
\sld
{\bf Usage}
  \begin{itemize}\isep
   \item{
Filters the ClassDoc passed.  If true is returned,
  the passed class will be included into the output.
  If false is returned, this document will not be
  included.
}%end item
  \end{itemize}
}
\end{itemize}
}%end item
\end{itemize}
}
}
}
\section{Classes}{
\startsection{Class}{Package}{l1}{%
{\small This class is used to manage the contents of a Java package.
  It accepts ClassDoc objects and examines them and groups them
  according to whether they are classes, interfaces, exceptions
  or errors.  The accumulated Vectors can then be processed to
  get to all of the elements of the package that fall into each
  catagory.}
\vskip .1in 
\startsubsubsection{Declaration}{
\fbox{\vbox{
\hbox{\vbox{\small public 
class 
Package}}
\noindent\hbox{\vbox{{\bf extends} java.lang.Object}}
}}}
\startsubsubsection{Constructors}{
\vskip -2em
\begin{itemize}
\item{\vskip -1.9ex 
\membername{Package}
{\tt public {\bf Package}( {\tt java.lang.String } {\bf pkg} )
\label{l6}\label{l7}}%end signature
\begin{itemize}
\sld
\item{
\sld
{\bf Usage}
  \begin{itemize}\isep
   \item{
Construct a new object corresponding to the passed package
  name.
}%end item
  \end{itemize}
}
\item{
\sld
{\bf Parameters}
\sld\isep
  \begin{itemize}
\sld\isep
   \item{
\sld
{\tt pkg} - the package name to use}
  \end{itemize}
}%end item
\end{itemize}
}%end item
\end{itemize}
}
\startsubsubsection{Methods}{
\vskip -2em
\begin{itemize}
\item{\vskip -1.9ex 
\membername{addElement}
{\tt public void {\bf addElement}( {\tt com.sun.javadoc.ClassDoc } {\bf cd} )
\label{l8}\label{l9}}%end signature
\begin{itemize}
\sld
\item{
\sld
{\bf Usage}
  \begin{itemize}\isep
   \item{
Adds a ClassDoc element to this package.
}%end item
  \end{itemize}
}
\item{
\sld
{\bf Parameters}
\sld\isep
  \begin{itemize}
\sld\isep
   \item{
\sld
{\tt cd} - the object to add to this package}
  \end{itemize}
}%end item
\end{itemize}
}%end item
\end{itemize}
}
}
\startsection{Class}{TableInfo}{l2}{%
{\small This class provides support for converting HTML tables into LaTeX tables.
  Some of the things {\bf NOT} implemented include the following:
  \begin{itemize}
  \item{\vskip -.8ex valign attributes are not procesed, but align= is.
  }
\item{\vskip -.8ex rowspan attributes are not processed, but colspan= is.
  }
\item{\vskip -.8ex the argument to border= in the table tag is not used to control line size
  }
\end{itemize}

  \mbox{}\newline

  Here is an example table.
  \bl 
  
% Table #1
\newlength{\tblaabcaaaw}
\setlength{\tblaabcaaaw}{0.3333333333333333\hsize}
\newlength{\tblaabcaabw}
\setlength{\tblaabcaabw}{0.3333333333333333\hsize}
\newlength{\tblaabcaacw}
\setlength{\tblaabcaacw}{0.3333333333333333\hsize}
\begin{tabular}{|p{\tblaabcaaaw}|p{\tblaabcaabw}|p{\tblaabcaacw}|}
\hline {\bf Column 1 Heading} & {\bf Column two heading} & {\bf Column three heading   } \\ \hline
{data} & \multicolumn{2}{p{\tblaabcaabw}|}{Span two columns   } \\ \hline
{{\it more data}} & \multicolumn{1}{r|}{right} & \multicolumn{1}{p{\tblaabcaacw}|}{left   } \\ \hline
\multicolumn{3}{|p{\hsize}|}{
% Table #2
\newlength{\tblaaccaaaw}
\setlength{\tblaaccaaaw}{0.3333333333333333\hsize}
\newlength{\tblaaccaabw}
\setlength{\tblaaccaabw}{0.3333333333333333\hsize}
\newlength{\tblaaccaacw}
\setlength{\tblaaccaacw}{0.3333333333333333\hsize}
\begin{tabular}{|p{\tblaaccaaaw}|p{\tblaaccaabw}|p{\tblaaccaacw}|}
\hline \multicolumn{3}{|p{\tblaaccaaaw}|}{\bf A nested table example   } \\ \hline
{\bf Column 1 Heading} & {\bf Column two heading} & {\bf Column three heading}    \\ \hline
{data} & \multicolumn{2}{p{\tblaaccaabw}|}{Span two columns}    \\ \hline
{{\it more data}} & \multicolumn{1}{r|}{right} & \multicolumn{1}{p{\tblaaccaacw}|}{left}    \\ \hline
{{\tt
\newline
\phantom{ }\phantom{ }1\newline
\phantom{ }\phantom{ }2\newline
\phantom{ }\phantom{ }3\newline
\phantom{ }\phantom{ }4\newline
\phantom{ }\phantom{ }}
}    & {{\tt
\newline
\phantom{ }\phantom{ }first\phantom{ }line\newline
\phantom{ }\phantom{ }second\phantom{ }line\newline
\phantom{ }\phantom{ }third\phantom{ }line\newline
\phantom{ }\phantom{ }fourth\phantom{ }line\newline
\phantom{ }\phantom{ }}
}    \\ \hline
\end{tabular}
   } \\ \hline
\end{tabular}
}
\vskip .1in 
\startsubsubsection{Declaration}{
\fbox{\vbox{
\hbox{\vbox{\small public 
class 
TableInfo}}
\noindent\hbox{\vbox{{\bf extends} java.lang.Object}}
}}}
\startsubsubsection{Constructors}{
\vskip -2em
\begin{itemize}
\item{\vskip -1.9ex 
\membername{TableInfo}
{\tt public {\bf TableInfo}( {\tt java.util.Properties } {\bf p},
{\tt java.lang.StringBuffer } {\bf ret},
{\tt java.lang.String } {\bf table},
{\tt int } {\bf off} )
\label{l10}\label{l11}}%end signature
\begin{itemize}
\sld
\item{
\sld
{\bf Usage}
  \begin{itemize}\isep
   \item{
Constructs a new table object and starts processing of the table by
  scanning the {\tt \textless table\textgreater } passed to count columns.
}%end item
  \end{itemize}
}
\item{
\sld
{\bf Parameters}
\sld\isep
  \begin{itemize}
\sld\isep
   \item{
\sld
{\tt p} - properties found on the {\tt \textless table\textgreater } tag}
   \item{
\sld
{\tt ret} - the result buffer that will contain the output}
   \item{
\sld
{\tt table} - the input string that has the entire table definition in it.}
   \item{
\sld
{\tt off} - the offset into {\tt \textless table\textgreater } where scanning should start}
  \end{itemize}
}%end item
\end{itemize}
}%end item
\end{itemize}
}
\startsubsubsection{Methods}{
\vskip -2em
\begin{itemize}
\item{\vskip -1.9ex 
\membername{endCol}
{\tt public void {\bf endCol}( {\tt java.lang.StringBuffer } {\bf ret} )
\label{l12}\label{l13}}%end signature
\begin{itemize}
\sld
\item{
\sld
{\bf Usage}
  \begin{itemize}\isep
   \item{
Ends the current column.
}%end item
  \end{itemize}
}
\item{
\sld
{\bf Parameters}
\sld\isep
  \begin{itemize}
\sld\isep
   \item{
\sld
{\tt ret} - the output buffer to put LaTeX into}
  \end{itemize}
}%end item
\end{itemize}
}%end item
\divideents{endRow}
\item{\vskip -1.9ex 
\membername{endRow}
{\tt public void {\bf endRow}( {\tt java.lang.StringBuffer } {\bf ret} )
\label{l14}\label{l15}}%end signature
\begin{itemize}
\sld
\item{
\sld
{\bf Usage}
  \begin{itemize}\isep
   \item{
Ends the current row.
}%end item
  \end{itemize}
}
\item{
\sld
{\bf Parameters}
\sld\isep
  \begin{itemize}
\sld\isep
   \item{
\sld
{\tt ret} - the output buffer to put LaTeX into}
  \end{itemize}
}%end item
\end{itemize}
}%end item
\divideents{endTable}
\item{\vskip -1.9ex 
\membername{endTable}
{\tt public void {\bf endTable}( {\tt java.lang.StringBuffer } {\bf ret} )
\label{l16}\label{l17}}%end signature
\begin{itemize}
\sld
\item{
\sld
{\bf Usage}
  \begin{itemize}\isep
   \item{
Ends the table, closing the last row as needed
}%end item
  \end{itemize}
}
\item{
\sld
{\bf Parameters}
\sld\isep
  \begin{itemize}
\sld\isep
   \item{
\sld
{\tt ret} - the output buffer to put LaTeX into}
  \end{itemize}
}%end item
\end{itemize}
}%end item
\divideents{startCol}
\item{\vskip -1.9ex 
\membername{startCol}
{\tt public void {\bf startCol}( {\tt java.lang.StringBuffer } {\bf ret},
{\tt java.util.Properties } {\bf p} )
\label{l18}\label{l19}}%end signature
\begin{itemize}
\sld
\item{
\sld
{\bf Usage}
  \begin{itemize}\isep
   \item{
Starts a new column, possibly closing the current column if needed
}%end item
  \end{itemize}
}
\item{
\sld
{\bf Parameters}
\sld\isep
  \begin{itemize}
\sld\isep
   \item{
\sld
{\tt ret} - the output buffer to put LaTeX into}
   \item{
\sld
{\tt p} - the properties from the {\tt \textless td\textgreater } tag}
  \end{itemize}
}%end item
\end{itemize}
}%end item
\divideents{startHeadCol}
\item{\vskip -1.9ex 
\membername{startHeadCol}
{\tt public void {\bf startHeadCol}( {\tt java.lang.StringBuffer } {\bf ret},
{\tt java.util.Properties } {\bf p} )
\label{l20}\label{l21}}%end signature
\begin{itemize}
\sld
\item{
\sld
{\bf Usage}
  \begin{itemize}\isep
   \item{
Starts a new Heading column, possibly closing the current column
  if needed.  A Heading column has a Bold Face font directive around
  it.
}%end item
  \end{itemize}
}
\item{
\sld
{\bf Parameters}
\sld\isep
  \begin{itemize}
\sld\isep
   \item{
\sld
{\tt ret} - the output buffer to put LaTeX into}
   \item{
\sld
{\tt p} - the properties from the {\tt \textless th\textgreater } tag}
  \end{itemize}
}%end item
\end{itemize}
}%end item
\divideents{startRow}
\item{\vskip -1.9ex 
\membername{startRow}
{\tt public void {\bf startRow}( {\tt java.lang.StringBuffer } {\bf ret},
{\tt java.util.Properties } {\bf p} )
\label{l22}\label{l23}}%end signature
\begin{itemize}
\sld
\item{
\sld
{\bf Usage}
  \begin{itemize}\isep
   \item{
Starts a new row, possibly closing the current row if needed
}%end item
  \end{itemize}
}
\item{
\sld
{\bf Parameters}
\sld\isep
  \begin{itemize}
\sld\isep
   \item{
\sld
{\tt ret} - the output buffer to put LaTeX into}
   \item{
\sld
{\tt p} - the properties from the {\tt \textless tr\textgreater } tag}
  \end{itemize}
}%end item
\end{itemize}
}%end item
\end{itemize}
}
}
\startsection{Class}{TexDoclet}{l3}{%
{\small This class provides a Java 2, {\tt javadoc} Doclet which generates
  a LaTeX2e document out of the java classes that it is used on.  This is
  convienent for creating printable documentation complete with cross reference
  information.
  \bl 
  Supported HTML tags within comments include the following
  \begin{itemize}

  \item[\textless dl\textgreater 
  ]{with the associated \textless dt\textgreater \textless dd\textgreater \textless $/$dl\textgreater  tags
  }
\item[\textless p\textgreater 
  ]{but not align=center...yet
  }
\item[\textless br\textgreater 
  ]{but not clear=xxx
  }
\item[\textless table\textgreater 
  ]{including all the associcated \textless td\textgreater \textless th\textgreater \textless tr\textgreater \textless $/$td\textgreater \textless $/$th\textgreater \textless $/$tr\textgreater 
  }
\item[\textless ol\textgreater 
  ]{ordered lists
  }
\item[\textless ul\textgreater 
  ]{unordered lists
  }
\item[\textless font\textgreater 
  ]{font coloring
  }
\item[\textless pre\textgreater 
  ]{preformatted text
  }
\item[\textless code\textgreater 
  ]{fixed point fonts
  }
\item[\textless i\textgreater 
  ]{italized fonts
  }
\item[\textless b\textgreater 
  ]{bold fonts
	}
\end{itemize}\bl  TexDoclet\refdefined{l24} 
 start\refdefined{l25} }
\vskip .1in 
\startsubsubsection{Declaration}{
\fbox{\vbox{
\hbox{\vbox{\small public 
class 
TexDoclet}}
\noindent\hbox{\vbox{{\bf extends} com.sun.javadoc.Doclet}}
}}}
\startsubsubsection{Fields}{
\begin{itemize}
\item{
public static PrintWriter os\begin{itemize}\item{\vskip -.9ex Writer for writing to output file}\end{itemize}
}
\end{itemize}
}
\startsubsubsection{Constructors}{
\vskip -2em
\begin{itemize}
\item{\vskip -1.9ex 
\membername{TexDoclet}
{\tt public {\bf TexDoclet}(  )
\label{l26}\label{l27}}%end signature
}%end item
\end{itemize}
}
\startsubsubsection{Methods}{
\vskip -2em
\begin{itemize}
\item{\vskip -1.9ex 
\membername{optionLength}
{\tt public static int {\bf optionLength}( {\tt java.lang.String } {\bf option} )
\label{l28}\label{l29}}%end signature
\begin{itemize}
\sld
\item{
\sld
{\bf Usage}
  \begin{itemize}\isep
   \item{
Returns how many arguments would be consumed if {\tt option}
  is a recognized option.
}%end item
  \end{itemize}
}
\item{
\sld
{\bf Parameters}
\sld\isep
  \begin{itemize}
\sld\isep
   \item{
\sld
{\tt option} - the option to check}
  \end{itemize}
}%end item
\end{itemize}
}%end item
\divideents{start}
\item{\vskip -1.9ex 
\membername{start}
{\tt public static boolean {\bf start}( {\tt com.sun.javadoc.RootDoc } {\bf root} )
\label{l30}\label{l31}}%end signature
\begin{itemize}
\sld
\item{
\sld
{\bf Usage}
  \begin{itemize}\isep
   \item{
Called by the framework to format the entire document
}%end item
  \end{itemize}
}
\item{
\sld
{\bf Parameters}
\sld\isep
  \begin{itemize}
\sld\isep
   \item{
\sld
{\tt root} - the root of the starting document}
  \end{itemize}
}%end item
\end{itemize}
}%end item
\divideents{validOptions}
\item{\vskip -1.9ex 
\membername{validOptions}
{\tt public static boolean {\bf validOptions}( {\tt java.lang.String [][]} {\bf args},
{\tt com.sun.javadoc.DocErrorReporter } {\bf err} )
\label{l32}\label{l33}}%end signature
\begin{itemize}
\sld
\item{
\sld
{\bf Usage}
  \begin{itemize}\isep
   \item{
Checks the passed options and their arguments for validity.
}%end item
  \end{itemize}
}
\item{
\sld
{\bf Parameters}
\sld\isep
  \begin{itemize}
\sld\isep
   \item{
\sld
{\tt args} - the arguments to check}
   \item{
\sld
{\tt err} - the interface to use for reporting errors}
  \end{itemize}
}%end item
\end{itemize}
}%end item
\end{itemize}
}
\startsubsubsection{Methods inherited from class {\tt com.sun.javadoc.Doclet}}{
\par{\small 
\refdefined{l34}\vskip -2em
\begin{itemize}
\item{\vskip -1.9ex 
\membername{languageVersion}
{\tt public static LanguageVersion {\bf languageVersion}(  )
}%end signature
}%end item
\divideents{optionLength}
\item{\vskip -1.9ex 
\membername{optionLength}
{\tt public static int {\bf optionLength}( {\tt java.lang.String } {\bf arg0} )
}%end signature
}%end item
\divideents{start}
\item{\vskip -1.9ex 
\membername{start}
{\tt public static boolean {\bf start}( {\tt com.sun.javadoc.RootDoc } {\bf arg0} )
}%end signature
}%end item
\divideents{validOptions}
\item{\vskip -1.9ex 
\membername{validOptions}
{\tt public static boolean {\bf validOptions}( {\tt java.lang.String [][]} {\bf arg0},
{\tt com.sun.javadoc.DocErrorReporter } {\bf arg1} )
}%end signature
}%end item
\end{itemize}
}}
}
}
}
\end{document}
