%% Cyberinfrastructure Shell (CIShell) Core Specification
%%
%% Copyright 2006,2007,2008 Indiana University
%%
%% Licensed under the Apache License, Version 2.0 (the "License");
%% you may not use this file except in compliance with the License.
%% You may obtain a copy of the License at
%%
%%     http://www.apache.org/licenses/LICENSE-2.0
%%
%% Unless required by applicable law or agreed to in writing, software
%% distributed under the License is distributed on an "AS IS" BASIS,
%% WITHOUT WARRANTIES OR CONDITIONS OF ANY KIND, either express or implied.
%% See the License for the specific language governing permissions and
%% limitations under the License.
%%
%

\chapter{Introduction}

The Cyberinfrastructure Shell (CIShell) is an open source, community-driven
platform for the integration and utilization of datasets, algorithms, tools, and
computing resources. It is built specifically to enable (1) algorithm developers
to write and disseminate their algorithms in their favorite programming language
while retaining their intellectual rights after distribution; (2) data holders to
easily disseminate their data for use by others; (3) application developers to
design applications from custom sets of algorithms and datasets that interoperate
seamlessly; and ultimately (4) end-users to use datasets and algorithms
effectively.

\section{Acknowledgements}

The Cyberinfrastructure Shell was designed and developed at the
Cyberinfrastructure for Network Science Center (CNSC) at Indiana University in
Bloomington, Indiana. The specification and API was designed and authored by
Bruce W. Herr II, but received input from all members of the CIShell team.
Important contributors from the CIShell team include Katy B\"{o}rner (director of
CNSC), Weixia Huang, Russell Duhon, Micah Linnemeier, and Timothy Kelley. Much of
the design of CIShell draws on previous work by Shashikant Penumarthy, Bruce W.
Herr II, and Katy B\"{o}rner on the Information Visualization Cyberinfrastructure
(IVC). Thanks go out to all those who have used or contributed to IVC, CIShell,
and the Network Workbench (the first project to use CIShell). Development of the
Cyberinfrastructure Shell was funded by grants two from the National Science
Foundation: NSF IIS-0238261 and NSF IIS-0513650.

\section{CIShell Platform Overview}

The CIShell Platform consists of Java interface definitions for algorithms, data,
services for algorithm developers, and services for application developers. Much
of the platform uses metadata and is fully defined.

This specification and associated Java API are released under the Apache 2.0
License.

%The next version should have this section.
%\section{What is New}

%This is the first release of the CIShell Platform Specification. Future
%versions will strive for backwards compatibility.

\section{Reader Level}

This specification is written for the following audiences:
\begin{itemize}
  \item Java algorithm developers
  \item Non-Java algorithm developers
  \item Framework and system service developers (system developers)
  \item Application developers building on CIShell
\end{itemize}

The CIShell Specification assumes that the reader has at least one year of
practical experience in writing Java programs. CIShell is built to run on the
OSGi Service Platform Release 4\footnote{http://www.osgi.org/Release4/Download}
and thus a working knowledge of OSGi is expected. OSGi (and thus CIShell) is
highly dynamic and must be taken into consideration when developing anything on
CIShell.

Non-Java algorithm developers may not need to know any Java and should be mainly
concerned with the metadata definitions for algorithms and data. They may also
need to be aware of OSGi and the other services CIShell provides, but more than
likely will not directly interact with them.

\section{Conventions and Terms}

In this specification, algorithms are referred to in three different contexts. An
abstract algorithm is the pure idea of the algorithm with no actual source code.
It is a series of steps sometimes put into pseudo-code and often published in
academic journals. An \class{Algorithm} with a capital A refers to the Java class
called Algorithm. And finally, an algorithm with a lowercase A refers to the
bundle of code and metadata that encompasses an algorithm written to work with
the CIShell Platform. This includes the implementation of
\class{AlgorithmFactory} and \class{Algorithm}, and the metadata, files, and
other code that go into an OSGi bundle.

All other conventions and terms are exactly the same as from OSGi's Core
Specification, section 1.4.

\section{Version Information}

This is the first release of the CIShell Platform Specification. All packages are
at 1.0 for this release. Subsequent releases may increase the version number of
specific packages if changes have been made.

\begin{table}[h!]
\begin{tabular}{l l l}
\textbf{Item} & \textbf{Package} & \textbf{Version} \\
Framework Specification & org.cishell.framework & Version 1.0 \\
Algorithm Specification & org.cishell.framework.algorithm & Version 1.0 \\
Data Specification & org.cishell.framework.data & Version 1.0 \\
User Adjustable Preferences Specification & org.cishell.framework.userprefs &
Version 1.0 \\
Data Conversion Service Specification & org.cishell.service.conversion &
Version 1.0 \\
GUI Builder Service Specification & org.cishell.service.guibuilder & Version
1.0 \\
Data Manager Application Service Specification & org.cishell.app.datamanager &
Version 1.0 \\
Scheduler Application Service Specification & org.cishell.app.scheduler &
Version 1.0 \\
\end{tabular}
\caption{Packages and Versions}
\label{table:packageVersions}
\end{table}