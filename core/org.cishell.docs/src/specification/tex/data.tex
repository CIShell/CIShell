\section{Data Specification}
\label{dataSpec}
\subsection*{\textit{Version 1.0}}
\subsection{Introduction}

Data to be operated on is passed around in \class{Data} objects which hold the
real data, the data's format, and its properties (metadata). The data can be any
Java \class{Object}. The format is a string which is either a full Java class or
a mime type if it is a \class{java.io.File}. The mime type corresponds to the
file's data type and has the following form - ``file:\textit{mime/type}''. Note
that if no official mime type is available for a file format, a made up one can
be used, but must still conform to how mime types are constructed. See RFCs 3023
and 4288 for more information on how to construct MIME types. Finally, the
properties help describe the data. The label to give the data, the parent
\class{Data} object from which it was derived from, and a coarse data type can
all be defined in the \class{Data}'s properties. See the \class{DataProperty}
interface definition for specific properties to use.
