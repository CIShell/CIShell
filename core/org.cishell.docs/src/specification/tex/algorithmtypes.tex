\section{Algorithm Type Specifications}

\subsection*{\textit{Version 1.0}}

\label{algConstraints}

\subsection{Base Algorithm Constraints}

All conformant algorithms regardless of type, must adhere to the following
constraints:

\subsubsection*{Required:}
\begin{itemize}
  \item The algorithm must be a conformant \class{AlgorithmFactory}
  implementation and properly registered as a service.
  \item The algorithm's service metadata must contain a legal ``service.pid''.
\end{itemize}

\subsubsection*{Optional:}
\begin{itemize}
  \item The algorithm's service metadata should have ``remoteable=true'' if it
  meets the requirements of a remoteable algorithm.
  \item The algorithm's service metadata should have a ``label'' which is a
  short human-readable name for the algorithm.
  \item The algorithm's service metadata should have a ``description''
  describing what the algorithm does in more detail.
  \item As much of the informational metadata as possible should be
  provided. This includes ``authors'', ``implementors'', ``integrators'',
  ``documentation\_url'', ``reference'', ``reference\_url'', and ``written\_in''.
\end{itemize}

\subsection{Standard Algorithms}

Standard CIShell algorithms are the algorithms that most end-users will
encounter. A standard algorithm has the following constraints:

\subsubsection*{Required:}
\begin{itemize}
  \item The algorithm must be a conformant \class{AlgorithmFactory}
  implementation and properly registered as a service.
  \item The algorithm's service metadata must contain a legal ``service.pid''.
  \item The algorithm's service metadata must have a ``label'' which is a
  short human-readable name for the algorithm. This is typically used to label
  an algorithm for an end-user to see.
  \item The algorithm's service metadata must have a ``description''
  describing what the algorithm does in more detail.
  \item The algorithm's service metadata must have a ``menu\_path'' which is
  simultaneously a classification and a location on a GUI's menubar to place
  the algorithm in. See section \ref{algMetaData} for how to format a
  ``menu\_path''.
\end{itemize}

\subsubsection*{Optional:}
\begin{itemize}
  
  \item The algorithm's service metadata should have ``remoteable=true'' if it
  meets the requirements of a remoteable algorithm.
  \item The algorithm's service metadata should have ``parentage=default'' if
  it wishes to use the default \class{Data} parenting scheme described in
  section \ref{algMetaData}.
  \item The algorithm's service metadata does not need to have a ``type'' set.
  \item As much of the informational metadata as possible should be
  provided. This includes ``authors'', ``implementors'', ``integrators'',
  ``documentation\_url'', ``reference'', ``reference\_url'', and ``written\_in''.
\end{itemize}


\subsection{Converter Algorithms}
\label{converterAlg}

A converter algorithm is a custom type of CIShell algorithm for converting data
of one type to another. Converters are typically leveraged by the
\class{DataConversionService} and are not used directly by end-users. A converter
algorithm has the following constraints:

\subsubsection*{Required:}
\begin{itemize}
  \item The algorithm must be a conformant \class{AlgorithmFactory}
  implementation and properly registered as a service.
  \item The algorithm must take in a single \class{Data} item and convert the
  item producing a single \class{Data} item. This must be reflected in the
  algorithm's service metadata where ``in\_data'' and ``out\_data'' have only
  one format each.
  \item The algorithm's service metadata must contain a legal ``service.pid''.
  \item The algorithm's service metadata must have ``type=converter''.
  \item The algorithm's service metadata must have ``conversion=lossy'' if
  data is lost during conversion or ``conversion=lossless'' if not.
\end{itemize}

\subsubsection*{Optional:}
\begin{itemize}
  \item The algorithm's service metadata should have ``remoteable=true'' if it
  meets the requirements of a remoteable algorithm.
  \item The algorithm's service metadata should have a ``label'' which is a
  short human-readable name for the converter, usually with the common name of
  the input data format and output data format.
  \item The algorithm's service metadata should have a ``description''
  describing the conversion in more detail, especially what data may be lossed
  if ``conversion=lossy''.
  \item The algorithm's service metadata should have ``implementers'' filled
  in accordingly.  
\end{itemize}

\subsection{Validator Algorithms}

A validator algorithm is a custom type of CIShell algorithm which checks either
an incoming or outgoing file to be sure it is of the type specified. This is
necessary due to the fact that one cannot simply assume based on the file
extension what type of file format the data is in. Checking the contents of the
file is necessary, especially in the case of multiple file formats for the same
file extension (e.g., XML). This type of algorithm is important for reliably
bringing in outside data and saving out data from CIShell. A validator algorithm
has the following constraints:

\subsubsection*{Required:}
\begin{itemize}
  \item The algorithm must be a conformant \class{AlgorithmFactory}
  implementation and properly registered as a service.
  \item The algorithm must take in a single \class{Data} item and validate the
  item producing a single \class{Data} item (with the same data, but changed
  format) or \class{null} if the file being validated is not of the right
  type. This must be reflected in the algorithm's service metadata where
  ``in\_data'' and ``out\_data'' have only one format each with one containing
  a ``file:'' format and the other a ``file-ext:'' depending on the direction
  of validation.
  \item The algorithm must not alter the data. Its only purpose is to validate
  the proposed incoming or outgoing file.
  \item The algorithm's service metadata must contain a legal ``service.pid''.
  \item The algorithm's service metadata must have ``type=validator''.
  \item The algorithm's service metadata must have a ``label'' which is the
  common name of the data format being validated.
\end{itemize}

\subsubsection*{Optional:}
\begin{itemize}
  \item The algorithm's service metadata should have ``remoteable=true'' if it
  meets the requirements of a remoteable algorithm.
  \item The algorithm's service metadata should have ``implementers'' filled
  in accordingly.
\end{itemize}
